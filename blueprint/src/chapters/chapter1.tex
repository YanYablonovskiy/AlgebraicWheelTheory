\section{Algebraic structures} 
Given a wheel $W$, certain sub-sets form familiar structures. 
\subsection{Induced semiring}
Consider the subset:
\[
\mathcal{R}_{W} := \{ w \in W | 0*w = 0 \}
\]
then $\mathcal{R}$ turns out to be a commutative unital semiring.
\begin{remark}
 For the purposes of this document, we assume semirings to be unital and commutative by default.
\end{remark}
Firstly,
\begin{definition}[\textbf{$\mathcal{R}_{W}$ is a commutative magma in * and +}]
  \label{def:R_w_magma}
  \lean{Wheel.instRCommMagma,Wheel.instAddCommMagma}
  \leanok
  Given a wheel $W$, then $\mathcal{R}_{W}$ is a commutative magma in both the wheel operations.
\end{definition}
This is primarily to address the `closure' of the algebraic operations in the sub-set. To achieve a semi-ring,
we further need two commutative monoids: one for each of the wheel operations. Having closure, what is left to prove is
a neutral element and associativity.
\begin{definition}[\textbf{$\mathcal{R}_{W}$ is a commutative monoid in * and +}]
  \label{def:R_w_monoid}
  \lean{Wheel.instRCommMonoid,Wheel.instAddCommMonoid}
  \leanok
  Given a wheel $W$, then $\mathcal{R}_{W}$ is a commutative monoid in both the wheel operations.
\end{definition}
Finally, the interaction between the two monoids must be distributive, and the semiring has been defined:
\begin{definition}[\textbf{$\mathcal{R}_{W}$ is a semiring in * and +}]
  \label{def:R_w_semiring}
  \lean{Wheel.instSemiRing,Wheel.toSemiring,Wheel.instLeftDistrib}
  Given a wheel $W$, then $\mathcal{R}_{W}$ is a semiring in * and +.
\end{definition}
\subsection{Induced commutative group}
In this section, consider the subset of a wheel $W$:
\[
  \mathcal{S}_{W} := \{ w \in W | 0*w = 0 \wedge 0*\backslash_{a}w = 0 \},
\]
note that $\mathcal{R}_{W}$ is a subset of this set. Once more, begining with the closure of the binary operation *
inherited from the wheel:
\begin{definition}[\textbf{$\mathcal{S}_{W}$ is a commutative magma in *}]
  \label{def:S_w_magma}
  \lean{Wheel.instSCommMagma}
  \leanok
  Given a wheel $W$, then $\mathcal{S}_{W}$ is a commutative magma in the * wheel operation.
\end{definition}
Followed by the commutative monoid:
\begin{definition}[\textbf{$\mathcal{S}_{W}$ is a commutative monoid in *}]
  \label{def:S_w_monoid}
  \lean{Wheel.instSCommMonoid}
  \leanok
  Given a wheel $W$, then $\mathcal{S}_{W}$ is a commutative monoid in the * wheel operation.
\end{definition}
And finally, the induced commutative group,
\begin{definition}[\textbf{$\mathcal{S}_{W}$ is a commutative group in *}]
  \label{def:S_w_group}
  \lean{Wheel.instCommGroup}
  \leanok
  Given a wheel $W$, then $\mathcal{S}_{W}$ is a commutative group in *.
\end{definition}
This shows that every element of $\mathcal{S}_{W}$ is $\backslash_{a}$-invertible. Furthermore, if an element of $\mathcal{R}_{W}$ 
is a $\backslash_{a}$-unit, then it is contained in $\mathcal{S}_{W}$:
\begin{proposition}
  \label{prop:S_w_of_wheel_invertible_R_w}
  \lean{Wheel.isUnit_wdiv_coe}
  \leanok
  For a wheel $W$, if $x \in \mathcal{R}_{W}$ is $\backslash_{a}$-invertible, then $x \in \mathcal{S}_{W}$.
\end{proposition}
\begin{proposition}
  \label{prop:wheel_unit_of_R_w_unit}
  \lean{Wheel.isRUnit_isUnit}
  \leanok
  For a wheel $W$, if $x \in \mathcal{R}_{W}$ is a unit, then is is also a unit in the original wheel.
\end{proposition}
\subsection{Involution Monoid}
An Involution is a monoid with a special involution operation , interacting in a particular way with the 
monoid operation.
\begin{definition}[\textbf{Involution Monoid}]
  \label{def:involution_monoid}
  \lean{InvolutionMonoid}
  \leanok
  Given a monoid $(M,*)$, $M$ is called an involution monoid for an involutive function $\star:M \rightarrow M$
  which skew-distributes over $*$. In other words, $(M,*)$ is a monoid, with an associated map $\star:M \rightarrow M$
  such that:
  \begin{enumerate}
    \item $\forall x \in M,$  $x \star \star = x$ \\
    \item $\forall x, y \in M, (x * y) \star = y \star  * \: x \star$. \\
  \end{enumerate}
  This is denoted as $(M,*,\star)$ or simply $(M,\star)$ if $*$ is inferrable from surrounding context.
\end{definition}
Several structures can be made into an involution monoid:
\begin{proposition}
  \label{prop:inv_monoid_of_abelian_group}
  \lean{InvolutionMonoid.instDivMonoid,InvolutionMonoid.group_example}
  \leanok
  Any abelian group is an involution monoid with $x\star = x^{-1}$.
\end{proposition}
\begin{proposition}
  \label{prop:inv_monoid_of_field}
  \lean{InvolutionMonoid.instField}
  \leanok
  Any field with decideable equality is an involution monoid with $x\star = x^{-1}$ for $x$ non-zero and $0$ otherwise.
\end{proposition}
\begin{proposition}
  \label{prop:sq_matrices_inv_monoid}
  \lean{InvolutionMonoid.instSqMatrix}
  \leanok
  The monoid of $n \times n$ $R$-valued matrices form an involution monoid with $x\star = x^{T}$ and $R$ a commutative ring.
\end{proposition}
Furthermore the notion of a $\star$-unit is defined on an involution monoid as a regular monoid unit, except that the two-sided inverse must
equal to the $\star$ operation.
\begin{definition}[\textbf{$\star$-unit}]
  \label{def:star_unit}
  \lean{InvolutionMonoid.StarUnit}
  \leanok
  An element of an involution monoid $x \in (M , \star)$ is a $\star$-unit if $x^{-1} = x\star$.
\end{definition}
\begin{definition}[\textbf{$\star$-invertible}]
  \label{def:star_invertibility}
  \lean{InvolutionMonoid.IsStarInv}
  \leanok
  An involution monoid $(M,\star)$ is $\star$-invertible if every unit is a $\star$-unit.
\end{definition}
The morphisms that respect such structures are similarly defined:
\begin{definition}[\textbf{Involution Monoid Homomorphism}]
  \label{def:involution_monoid_morphism}
  \lean{InvolutionMonoidHom}
  \leanok
  If there are two involution monoids, $(M_{1},\star_{1})$ and $(M_{2},\star_{2})$, a monoid homomorphism $f : M_{1} \rightarrow M_{2}$ 
  is an involution monoid homomorphism if $\forall x ∈ M_{1}, f (x\star_{1}) = f (x)\star_{2}$.
\end{definition}
\subsection{Commutative monoid to Involution monoid}
It is possible, given a commutative monoid $M$, to construct an involution monoid contained within $M$. To begin with , an Involution
monoid may be defined on $M \times M$, with $(x_{1},x_{2})\star = (x_{2},x_{1})$.
\begin{definition}
  \label{def:involution_monoid_prod_space}
  \lean{instProdStar,instProdInvMon}
  \leanok
  Given a commutative monoid $M$, there is an involution monoid on $M \times M$ with $(x_{1},x_{2})\star = (x_{2},x_{1})$, and 
  the usual pointwise monoid operation taken from $M$.
\end{definition}
Define an equivalence relation on $M \times M$:
\begin{definition}
  \label{def:involution_monoid_setoid}
  \lean{equiv_rel,instProdSetoid}
  \leanok
  $(x,y) \equiv (x',y')$ iff $\exists s_{1}, s_{2} \in M,(s_{1},s_{1})(x,y) = (s_{2},s_{2})(x',y')$
\end{definition}
The equivalence relation gives rise to the corresponding quotient space:
\begin{definition}
  \label{def:MStar}
  \lean{MStar}
  \leanok
  Denote the quotient space $M^{*}_{X} := M \times M / \equiv$.
\end{definition}
\subsection{Wheel Module}
\begin{definition}
  \label{def:wheel_module}
  \lean{WheelModule}
  \leanok
  Let $W$ be a wheel. A $W$-module, or WheelModule, is a commutative monoid
  $(M,0,+)$ with multiplication by W-elements defined (formally, a function
  $(H × M) → M$ written $(x,m) → x • m$) such that for any $x,x' ∈ W$, $m,m' ∈ M$
  the following hold:
  \begin{enumerate}
  \item $(xx') • m = x • (x' • m)$ \\
  \item $1 • m = m$ \\
  \item $(x + x') • m + 0 • m  = x • m + x' • m$ \\
  \item $x • (m + m') + x • 0 = x • m + x • m'$ \\
  \item $(\backslash_{a} x) • m + m' + x • 0 = \backslash_{a} x • (m + x • m')$ \\
  \item $(\backslash_{a} x) • m + m' + x • 0 = \backslash_{a} x • (m + x • m')$ \\
  \item $x • (m + 0 • m') = x • m + 0 • m'$ \\
  \item $m + ( \backslash_{a} 0 ) • 0 = ( \backslash_{a} 0 ) • 0$ 
  \end{enumerate}
\end{definition}