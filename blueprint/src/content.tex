% In this file you should put the actual content of the blueprint.
% It will be used both by the web and the print version.
% It should *not* include the \begin{document}
%
% If you want to split the blueprint content into several files then
% the current file can be a simple sequence of \input. Otherwise It
% can start with a \section or \chapter for instance.
\chapter{Introduction}
\section{Definition of a Wheel}
Agebraic wheels are structures generalising a commutative semiring, attempting to make sense of  `division' by zero. \par
Loosely speaking, given a semiring $R$ and it's associated monoids, one may extend the semi-ring in a variety of well-known ways. Considering an additive inverse extends a commutative semiring, to a structure with a given name: a commutative ring, and attempting the same succesfully for the multiplicative monoid yields a field. \par
Working backwards, given a set $M$ with two monoids -- one in additive notation and one in multiplicative. The path to obtain a semiring is clear,  a wheel however generalises the semiring by removing the usual distributivity and defines a new unary map $wDiv$.
\begin{definition}[\textbf{Wheel}]
  \label{def:Wheel}
  \lean{Wheel}
  \leanok
A Wheel $W$ is an algebraic structure which has two binary operations $(+,*)$, like a ring.
Similarly to a commutative ring, a Wheel is a commutative monoid in both operations. Additionally,
there is a multiplicative unary map $wDiv$  which is an involution, as well as a few idiosyncratic
properties in the interactions of the $+$,$*$ and $wDiv$.
  \begin{enumerate}
  \item Involution: $\forall w \in W, wDiv(wDiv(w)) = w$ 
  \item Multiplicative automophism: $\forall w, v \in W, wDiv(wv) = wDiv(w)wDiv(v)$
  \item Right distributivity rule 1: $\forall w, v, u \in W, (w + v)u + 0u= wu + vu$
  \item Right distributivity rule 2: $\forall w, v, u \in W, (w + 0v)u + 0u= wu + 0v$
  \item Right wDiv distributivity: $\forall w, v, u \in W, (w + uv)wDiv(u) = wDiv(u)+ v+0u$
  \item Division by 0: $\forall w \in W, 0Div(0) + w = 0Div(0)$
  \item Zero squared: $0*0 = 0$
  \item Division rule: $\forall w, v \in W, wDiv(w + 0v) = wDiv(w) +0v$
  \end{enumerate}
\end{definition}
Whenever not specified, the notation for the monoids is assumed to be $(+,*)$ with neutral elements $0$ and $1$ respectively.
\subsection{Basic results}
Here we collate some very simple propositions that are straightforward given the Wheel definition. These are designed to be auxillary 
and thus somewhat assorted and perhaps trivial, however mechanisation demands specification of what is typically deemed trivial.\par
Define the notation $\backslash_{a}  := wDiv $ for brevity.
\begin{proposition}[\textbf{Unit preserving}]
\label{prop:wDiv_one}
\lean{Wheel.wdiv_one}
\leanok
Given a Wheel $W$ , then $\backslash_{a} \backslash_{a} 1 = 1$  where $1$ is the neutral element of the multiplicative commutative monoid.
\end{proposition}
\begin{proposition}
\label{prop:zero_mul_add}
\lean{Wheel.zero_mul_add}
\leanok
Given a Wheel $W$ and any two elements $a,b \in W$, then:
\[
0*a + 0*b = 0*a*b
\]
\end{proposition}
\begin{proposition}
\label{prop:zero_wdiv_mul}
\lean{Wheel.zero_wdiv_mul}
\leanok
Given a Wheel $W$ and any element $a \in W$, then:
\[
(0* \backslash_{a}  0)*a = 0* \backslash_{a}  0
\]
\end{proposition}
\begin{proposition}[\textbf{Dividing by self}]
\label{prop:wdiv_selfl}
\lean{Wheel.wdiv_self}
\leanok
Given a Wheel $W$ and any element $a \in W$, then:
\[
a*\backslash_{a}  a = 1 + 0*(a*\backslash_{a}  a)
\]
\end{proposition}
\begin{proposition}[\textbf{Right cancellation}]
\label{prop:wdiv_right_cancell}
\lean{Wheel.wdiv_right_cancel'}
\leanok
Given a Wheel $W$ and any elements $a,b,c \in W$ such that $a*c = b*c$, then:
\[
a + 0*c*\backslash_{a}  c = b + 0*c*\backslash_{a} c
\]
\end{proposition}
\begin{proposition}[\textbf{Monoid Automorphism}]
\label{prop:wDivMonAuto}
\lean{Wheel.toMonoidHom}
\leanok
Given a Wheel $W$ ,  $wDiv$ is a monoid automorphism for $(1,*)$.
\end{proposition}
\chapter{References}
[1] JESPER CARLSTRÖM. “Wheels – on division by zero”. In: Mathematical Structures in Computer Science 14.1 (2004), pp. 143–184. doi: 10.1017/S0960129503004110.