% In this file you should put the actual content of the blueprint.
% It will be used both by the web and the print version.
% It should *not* include the \begin{document}
%
% If you want to split the blueprint content into several files then
% the current file can be a simple sequence of \input. Otherwise It
% can start with a \section or \chapter for instance.
\chapter{Introduction}
\section{Definition of a Wheel}
Algebraic wheels are structures generalising a commutative semiring, attempting to make sense of  `division' by zero. \par
Loosely speaking, given a semiring $R$ and it's associated monoids, one may extend the semi-ring in a variety of well-known ways. Considering an additive inverse extends a commutative semiring, to a structure with a given name: a commutative ring, and attempting the same succesfully for the multiplicative monoid yields a field. \par
Working backwards, given a set $M$ with two monoids -- one in additive notation and one in multiplicative. The path to obtain a semiring is clear,  a wheel however generalises the semiring by removing the usual distributivity and defines a new unary map $wDiv$.
\begin{definition}[\textbf{Wheel}]
  \label{def:Wheel}
  \lean{Wheel}
  \leanok
A Wheel $W$ is an algebraic structure which has two binary operations $(+,*)$, like a ring.
Similarly to a commutative ring, a Wheel is a commutative monoid in both operations. Additionally,
there is a multiplicative unary map $wDiv$  which is an involution, as well as a few idiosyncratic
properties in the interactions of the $+$,$*$ and $wDiv$.
  \begin{enumerate}
  \item Involution: $\forall w \in W, wDiv(wDiv(w)) = w$ 
  \item Multiplicative automophism: $\forall w, v \in W, wDiv(wv) = wDiv(w)wDiv(v)$
  \item Right distributivity rule 1: $\forall w, v, u \in W, (w + v)u + 0u= wu + vu$
  \item Right distributivity rule 2: $\forall w, v, u \in W, (w + 0v)u + 0u= wu + 0v$
  \item Right wDiv distributivity: $\forall w, v, u \in W, (w + uv)wDiv(u) = wDiv(u)+ v+0u$
  \item Division by 0: $\forall w \in W, 0Div(0) + w = 0Div(0)$
  \item Zero squared: $0*0 = 0$
  \item Division rule: $\forall w, v \in W, wDiv(w + 0v) = wDiv(w) +0v$
  \end{enumerate}
\end{definition}
Whenever not specified, the notation for the monoids is assumed to be $(+,*)$ with neutral elements $0$ and $1$ respectively.
\subsection{Basic results}
Here we collate some very simple propositions that are straightforward given the Wheel definition. These are designed to be auxillary 
and thus somewhat assorted and perhaps trivial, however mechanisation demands specification of what is typically deemed trivial.\par
Define the notation $\backslash_{a}  := wDiv $ for brevity.
\begin{proposition}[\textbf{Unit preserving}]
\label{prop:wDiv_one}
\lean{Wheel.wdiv_one}
\leanok
Given a Wheel $W$ , then $\backslash_{a} 1 = 1$  where $1$ is the neutral element of the multiplicative commutative monoid.
\end{proposition}
\begin{proposition}
\label{prop:zero_mul_add}
\lean{Wheel.zero_mul_add}
\leanok
Given a Wheel $W$ and any two elements $a,b \in W$, then:
\[
0a + 0b = 0ab
\]
\end{proposition}
\begin{proposition}
\label{prop:zero_wdiv_mul}
\lean{Wheel.zero_wdiv_mul}
\leanok
Given a Wheel $W$ and any element $a \in W$, then:
\[
(0 \backslash_{a}  0)a = 0 \backslash_{a}  0
\]
\end{proposition}
\begin{proposition}[\textbf{Dividing by self}]
\label{prop:wdiv_selfl}
\lean{Wheel.wdiv_self}
\leanok
Given a Wheel $W$ and any element $a \in W$, then:
\[
a\backslash_{a}  a = 1 + 0(a\backslash_{a}  a)
\]
\end{proposition}
\begin{proposition}[\textbf{Right cancellation}]
\label{prop:wdiv_right_cancell}
\lean{Wheel.wdiv_right_cancel'}
\leanok
Given a Wheel $W$ and any elements $a,b,c \in W$ such that $ac = bc$, then:
\[
a + 0c\backslash_{a} c = b + 0c\backslash_{a} c
\]
\end{proposition}
\begin{proposition}[\textbf{Monoid Automorphism}]
\label{prop:wDivMonAuto}
\lean{Wheel.toMonoidHom}
\leanok
Given a Wheel $W$ ,  $wDiv$ is a monoid automorphism for $(1,*)$.
\end{proposition}
\subsection{Unital interactions}
This section examines how a Wheel $W$ behaves when an element $x \in W$ happens to be a unit in the multiplicative monoid.
\begin{proposition}
  \label{prop:wdiv_inv_add_div}
  \lean{Wheel.isUnit_add_eq_div_add}
  \leanok
  Given a Wheel $W$, and $x \in W$ a unit in the multiplicative monoid of $W$, then the unit and self Wheel division are 
  related by:
  \begin{equation}
    x^{-1} + 0\backslash_{a}x = \backslash_{a}x + 0x^{-1}
  \end{equation}
  where $x^{-1} \in W$ is the associated two-sided multiplicative inverse of the unit $x$.
\end{proposition} 
\begin{proposition}
  \label{prop:isUnit_zero_eq_div_mul_add}
  \lean{Wheel.isUnit_zero_eq_div_mul_add}
  \leanok
  Given a Wheel $W$, and $x \in W$ a unit in the multiplicative monoid of $W$, then zero enjoys the following identity: 
  \begin{equation}
    0\backslash_{a}x + 0\backslash_{a}x^{-1} = 0
  \end{equation}
  where $x^{-1} \in W$ is the associated two-sided multiplicative inverse of the unit $x$.
\end{proposition} 
\begin{proposition}
  \label{prop:isUnit_inv_eq_div_add}
  \lean{Wheel.isUnit_inv_eq_div_add}
  \leanok
  Given a Wheel $W$, and $x \in W$ a unit in the multiplicative monoid of $W$, then: 
  \begin{equation}
    x^{-1} = \backslash_{a}x + 0x^{-1}\backslash_{a}x^{-1}
  \end{equation}
  where $x^{-1} \in W$ is the associated two-sided multiplicative inverse of the unit $x$.
\end{proposition} 
\begin{proposition}
  \label{prop:isUnit_div_eq_inv_add}
  \lean{Wheel.isUnit_div_eq_inv_add}
  \leanok
  Given a Wheel $W$, and $x \in W$ a unit in the multiplicative monoid of $W$, then: 
  \begin{equation}
    \backslash_{a}x = x^{-1} + 0x\backslash_{a}x
  \end{equation}
  where $x^{-1} \in W$ is the associated two-sided multiplicative inverse of the unit $x$.
\end{proposition}
\subsection{The trivial wheel}
Given some set $\beta$, if there exist two commutative monoids which can be defined on this set, assume one is taken
with additive notation and the other with multiplicative. Then there exists 
a wheel for $\beta$ : the trivial wheel containing the multiplicative unit.
\begin{proposition}
  \lean{Wheel.Trivial.instTrivWheel}
  \leanok
  Suppose there is a set $\beta$ and there exist two commutative monoids which can be defined on this set, in additive
  and multiplicative notation. Then $\{ 1 \}$ is a Wheel with wheel division being the identity map.
\end{proposition}
If given a wheel $W$, one may ask if it is the trivial wheel. This may be answered by comparing any two values out of
$0,1,\backslash_{a} 0, 0 \backslash_{a} 0$ as below:
\begin{proposition}
\lean{Wheel.Trivial.triv_tfae}
\leanok
  If any two of the elements $0,1,\backslash_{a} 0$ and $0 \backslash_{a} 0$ are equal in a
wheel $W$, then $W$ is trivial.
\end{proposition} 
\section{Algebraic structures} 
Given a wheel $W$, certain sub-sets form familiar structures. 
\subsection{Induced semiring}
Consider the subset:
\[
\mathcal{R}_{W} := \{ w \in W | 0*w = 0 \}
\]
then $\mathcal{R}$ turns out to be a commutative unital semiring.
\begin{remark}
 For the purposes of this document, we assume semirings to be unital and commutative by default.
\end{remark}
Firstly,
\begin{definition}[\textbf{$\mathcal{R}_{W}$ is a commutative magma in * and +}]
  \lean{Wheel.instRCommMagma,Wheel.instAddCommMagma}
  \leanok
  Given a wheel $W$, then $\mathcal{R}_{W}$ is a commutative magma in both the wheel operations.
\end{definition}
This is primarily to address the `closure' of the algebraic operations in the sub-set. To achieve a semi-ring,
we further need two commutative monoids: one for each of the wheel operations. Having closure, what is left to prove is
a neutral element and associativity.
\begin{definition}[\textbf{$\mathcal{R}_{W}$ is a commutative monoid in * and +}]
  \lean{Wheel.instRCommMonoid,Wheel.instAddCommMonoid}
  \leanok
  Given a wheel $W$, then $\mathcal{R}_{W}$ is a commutative monoid in both the wheel operations.
\end{definition}
Finally, the interaction between the two monoids must be distributive, and the semiring has been defined:
\begin{definition}[\textbf{$\mathcal{R}_{W}$ is a semiring in * and +}]
  \lean{Wheel.instSemiRing,Wheel.toSemiring,Wheel.instLeftDistrib}
  Given a wheel $W$, then $\mathcal{R}_{W}$ is a semiring in * and +.
\end{definition}
\subsection{Induced commutative group}
In this section, consider the subset of a wheel $W$:
\[
  \mathcal{S}_{W} := \{ w \in W | 0*w = 0 \wedge 0*\backslash_{a}w = 0 \},
\]
note that $\mathcal{R}_{W}$ is a subset of this set. Once more, begining with the closure of the binary operation *
inherited from the wheel:
\begin{definition}[\textbf{$\mathcal{S}_{W}$ is a commutative magma in *}]
  \lean{Wheel.instSCommMagma}
  \leanok
  Given a wheel $W$, then $\mathcal{S}_{W}$ is a commutative magma in the * wheel operation.
\end{definition}
Followed by the commutative monoid:
\begin{definition}[\textbf{$\mathcal{S}_{W}$ is a commutative monoid in *}]
  \lean{Wheel.instSCommMonoid}
  \leanok
  Given a wheel $W$, then $\mathcal{S}_{W}$ is a commutative monoid in the * wheel operation.
\end{definition}
And finally, the induced commutative group,
\begin{definition}[\textbf{$\mathcal{S}_{W}$ is a commutative group in *}]
  \lean{Wheel.instCommGroup}
  \leanok
  Given a wheel $W$, then $\mathcal{S}_{W}$ is a commutative group in *.
\end{definition}
This shows that every element of $\mathcal{S}_{W}$ is $\backslash_{a}$-invertible. Furthermore, if an element of $\mathcal{R}_{W}$ 
is a $\backslash_{a}$-unit, then it is contained in $\mathcal{S}_{W}$:
\begin{proposition}
  \lean{Wheel.isUnit_wdiv_coe}
  \leanok
  For a wheel $W$, if $x \in \mathcal{R}_{W}$ is $\backslash_{a}$-invertible, then $x \in \mathcal{S}_{W}$.
\end{proposition}
\begin{proposition}
  \lean{Wheel.isRUnit_isUnit}
  \leanok
  For a wheel $W$, if $x \in \mathcal{R}_{W}$ is a unit, then is is also a unit in the original wheel.
\end{proposition}
\subsection{Involution Monoid}
An Involution is a monoid with a special involution operation , interacting in a particular way with the 
monoid operation.
\begin{definition}[\textbf{Involution Monoid}]
  \lean{InvolutionMonoid}
  \leanok
  Given a monoid $(M,*)$, $M$ is called an involution monoid for an involutive function $\star:M \rightarrow M$
  which skew-distributes over $*$. In other words, $(M,*)$ is a monoid, with an associated map $\star:M \rightarrow M$
  such that:
  \begin{enumerate}
    \item $\forall x \in M,$  $x \star \star = x$ \\
    \item $\forall x, y \in M, (x * y) \star = y \star  * \: x \star$. \\
  \end{enumerate}
  This is denoted as $(M,*,\star)$ or simply $(M,\star)$ if $*$ is inferrable from surrounding context.
\end{definition}
Several structures can be made into an involution monoid:
\begin{proposition}
  \lean{InvolutionMonoid.instDivMonoid,InvolutionMonoid.group_example}
  \leanok
  Any abelian group is an involution monoid with $x\star = x^{-1}$.
\end{proposition}
\begin{proposition}
  \lean{InvolutionMonoid.instField}
  \leanok
  Any field with decideable equality is an involution monoid with $x\star = x^{-1}$ for $x$ non-zero and $0$ otherwise.
\end{proposition}
\begin{proposition}
  \lean{InvolutionMonoid.instSqMatrix}
  \leanok
  The monoid of $n \times n$ $R$-valued matrices form an involution monoid with $x\star = x^{T}$ and $R$ a commutative ring.
\end{proposition}
Furthermore the notion of a $\star$-unit is defined on an involution monoid as a regular monoid unit, except that the two-sided inverse must
equal to the $\star$ operation.
\begin{definition}[\textbf{$\star$-unit}]
  \lean{InvolutionMonoid.StarUnit}
  \leanok
  An element of an involution monoid $x \in (M , \star)$ is a $\star$-unit if $x^{-1} = x\star$.
\end{definition}
\begin{definition}[\textbf{$\star$-invertible}]
  \lean{InvolutionMonoid.IsStarInv}
  \leanok
  An involution monoid $(M,\star)$ is $\star$-invertible if every unit is a $\star$-unit.
\end{definition}
The morphisms that respect such structures are similarly defined:
\begin{definition}[\textbf{Involution Monoid Homomorphism}]
  \lean{InvolutionMonoidHom}
  \leanok
  If there are two involution monoids, $(M_{1},\star_{1})$ and $(M_{2},\star_{2})$, a monoid homomorphism $f : M_{1} \rightarrow M_{2}$ 
  is an involution monoid homomorphism if $\forall x ∈ M_{1}, f (x\star_{1}) = f (x)\star_{2}$.
\end{definition}
\chapter{References}
[1] JESPER CARLSTRÖM. “Wheels – on division by zero”. In: Mathematical Structures in Computer Science 14.1 (2004), pp. 143–184. doi: 10.1017/S0960129503004110.